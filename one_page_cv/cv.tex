% FortySecondsCV LaTeX template
% Copyright © 2019 René Wirnata <rene.wirnata@pandascience.net>
% Licensed under the 3-Clause BSD License. See LICENSE file for details.
%
% Attributions
% ------------
% * fortysecondscv is based on the twentysecondcv class by Carmine Spagnuolo 
%   (cspagnuolo@unisa.it), released under the MIT license and available under
%   https://github.com/spagnuolocarmine/TwentySecondsCurriculumVitae-LaTex
% * further attributions are indicated immediately before corresponding code


%-------------------------------------------------------------------------------
%                             ADDITIONAL PACKAGES
%-------------------------------------------------------------------------------
\documentclass[
  a4paper, 
%   showframes,
%   vline=2.2em,
%   maincolor=cvgreen,
%   sectioncolor=red,
%   subsectioncolor=orange,
%   itemtextcolor=black!80,
%   sidebarwidth=0.4\paperwidth,
%   topbottommargin=0.03\paperheight,
%   leftrightmargin=20pt,
%   proilepicsize=4.5cm,
]{fortysecondscv}

% improve word spacing and hyphenation
\usepackage{microtype}
\usepackage{ragged2e}

% take care of proper font encoding
\ifxetexorluatex
	\usepackage{fontspec}
	\defaultfontfeatures{Ligatures=TeX}
% \newfontfamily\headingfont[Path = fonts/]{segoeuib.ttf} % local font
\else
	\usepackage[utf8]{inputenc}
	\usepackage[T1]{fontenc}
% \usepackage[sfdefault]{noto} % use noto google font
\fi

% enable mathematical syntax for some symbols like \varnothing
\usepackage{amssymb}

% bubble diagram configuration
\usepackage{smartdiagram}
\smartdiagramset{
  % defaut font size is \large, so adjust to harmonize with sidebar layout
  bubble center node font = \footnotesize,
  bubble node font = \footnotesize,
  % default: 4cm/2.5cm; make minimum diameter relative to sidebar size
  bubble center node size = 0.4\sidebartextwidth,
  bubble node size = 0.25\sidebartextwidth,
  distance center/other bubbles = 1.5em,
  % set center bubble color
  bubble center node color = maincolor!70,
  % define the list of colors usable in the diagram
  set color list = {maincolor!10, maincolor!40,
  maincolor!20, maincolor!60, maincolor!35},
  % sets the opacity at which the bubbles are shown
  bubble fill opacity = 0.8,
}


%-------------------------------------------------------------------------------
%                            PERSONAL INFORMATION
%-------------------------------------------------------------------------------
%% mandatory information
% your name
\cvname{Siddhartha}
% job title/career
\cvjobtitle{FPGAs / Machine Learning\\[0.2em] Engineer}

%% optional information
% profile picture
\cvprofilepic{pics/profile_compressed.jpg}

% short address/location, use \newline if more than 1 line is required
\cvaddress{Singapore}
% phone number
% \cvphone{+61 411 77 1811}
% personal website
\cvsite{https://sidmontu.github.io}
% email address
\cvmail{sidmontu@gmail.com}
% github
\cvgithub{https://github.com/sidmontu/}{Github}
% linkedin 
\cvlinkedin{https://www.linkedin.com/in/siddhartha-63249064/}{LinkedIn}
% google scholar 
% \cvscholar{https://scholar.google.com/citations?user=tRI56rsAAAAJ}{Google Scholar}

%-------------------------------------------------------------------------------
%                              SIDEBAR 1st PAGE
%-------------------------------------------------------------------------------
% add more profile sections to sidebar on first page
\addtofrontsidebar{
	% include gosquare national flags from https://github.com/gosquared/flags;
	% naming according to ISO 3166-1 alpha-2 country codes
	\graphicspath{{pics/flags/}}

    \profilesection{Personal Statement}
	\aboutme{
        Computer engineer/researcher with a strong academic track record
        looking for a fresh challenge. I am especially interested in hardware
        design and/or machine-learning related projects, as I have relevant
        experience in both areas. Lifelong learner open to picking up new
        technology stacks when needed, and ready to work hard towards ambitious
        targets.
	}

	% social network accounts incl. proper hyperlinks
% 	\profilesection{Social Network}
% 		\begin{icontable}{2.5em}{1em}
% 		    % overleaf still not supports Academicons and FontAwesome5 for XeLaTeX, which contain the overleaf logl...unbelievable...
% 			\social{\faGraduationCap}
% 				{https://scholar.google.com/citations?user=tRI56rsAAAAJ/}
% 				{Google Scholar}
% 			\social{\faGithub}
% 				{https://github.com/sidmontu/}
% 				{Github}
% 			\social{\faGlobe}
% 				{https://sidmontu.github.io/}
% 				{Personal Website}
% 		\end{icontable}

%	\profilesection{Languages}
%		\pointskill{\flag{CN.png}}{Chinese}{5}
%		\pointskill{\flag{DE.png}}{German}{3}
%  	\pointskill{\flag{GB.png}}{English}{3}
%  	\pointskill{\flag{FR.png}}{French}{3}

	\profilesection{Hardware}
        \skill{\faUniversity}{Strong foundational experience}
        \skill{\faKeyboard}{Proficient with Verilog}
        \skill{\faWrench}{Familiar with FPGA tools}
        \skill{\faTerminal}{VivadoHLS / PYNQ Framework}
        \skill{\faWifi}{RF communication fundamentals}
        % \skill{\faLightbulb}{Learning SystemVerilog/UVM}

	\profilesection{Software}
        \skill{\faLinux}{Experienced UNIX user, comfortable with command-line tools.}
        \skill{\faCuttlefish}{Experienced C/C++ programmer}
        \skill{\faPython}{Strong Python developer}
        \skill{\faCogs}{Tensorflow/PyTorch machine learning frameworks}
        \skill{\faChartBar}{Data Analysis using R}
        \skill{\faFont}{Typesetting with \LaTeX}
        \skill{\faCodeBranch}{Version control, {\it e.g.} \texttt{git}}
}


%-------------------------------------------------------------------------------
%                              SIDEBAR 2nd PAGE
%-------------------------------------------------------------------------------
\addtobacksidebar{
	
	\profilesection{Diagrams}
	\chartlabel{Bubble Diagram}
	\begin{figure}\centering
		\smartdiagram[bubble diagram]{
			\textcolor{white}{\textbf{Being a}} \\ 
			\textcolor{white}{\textbf{Panda}}, % center bubble	
			\textcolor{black!90}{Eating},
			\textcolor{black!90}{Sleeping},
			\textcolor{black!90}{Rolling},
			\textcolor{black!90}{Playing},
			\textcolor{black!90}{Chilling}
		}
	\end{figure}

	\chartlabel{Wheel Chart}

	\wheelchart{4em}{2em}{%
  	20/3em/maincolor!50/Chill,
  	15/3em/maincolor!15/Play,
  	30/4em/maincolor!40/Sleep,
  	20/3em/maincolor!20/Eat
	}

	\profilesection{Barskills}
	\barskill{\faSkyatlas}{Wearing asian rice hats}{60}
	\barskill{\faImage}{Playing Chess}{30}
	\barskill{\faMusic}{Playing the bamboo flute}{50}

	\profilesection{Memberships}
	\begin{memberships}
		\membership[4em]{pics/logo.png}{PandaScience.net}
		\membership[4em]{pics/logo.png}{Some long text spanning over more than
			only one line}
		\membership[4em]{pics/logo.png}{\rule{\linewidth}{1pt}}
	\end{memberships}
}


%-------------------------------------------------------------------------------
%                         TABLE ENTRIES RIGHT COLUMN
%-------------------------------------------------------------------------------
\begin{document}

\makefrontsidebar

\cvsection{Work Experience}
\vspace{-0.1in}
\begin{cvtable}[3]
	\cvitem{Oct 2017 - Dec 2019}{Postdoctoral Research Associate}{University of Sydney}{
        \textit{Main Project: High-speed machine learning for RF applications}
        \\ This project explored the feasibility of applying and implementing
        deep learning models on FPGAs for doing \textit{real-time}
        radio-frequency spectral prediction. I served as the lead technical
        engineer on this project, and built a software-framework for automating
        the end-to-end flow of modeling, training, and implementing
        convolutional neural networks on FPGAs. Topics/technology-stacks learnt
        during the course of the project: Python, Tensorflow/Tensorpack,
        VivadoHLS, low-precision neural networks. \\
        \textit{Other roles}: Project supervision / guidance to undergraduate
        students doing their final-year projects, involvement in other research
        projects in the lab.
    }
	\cvitem{Jun -- Oct 2012}{Undergraduate Research Assistant}{Imperial College London}{
        Under the University Research Opportunities Program (UROP), I embarked
        on a summer project under the supervision of
        \href{https://nachiket.github.io}{Nachiket Kapre}, which eventually
        served as a foundation to my PhD research. The work produced during
        this stint was published as a short paper in the 2014 IEEE
        Field-Programmable Custom Computing Machines conference proceedings.
    }
\end{cvtable}


\cvsection{Education}
\begin{cvtable}[1.5]
	\cvitem{2013 -- 2019}{Doctor of Philosophy}{Nanyang Technological University, Singapore}{
        \textit{Dissertation: Dataflow Optimized Overlays for FPGAs} \\
        This thesis introduces Dataflow Coprocessor Overlay (DaCO), a token
        dataflow overlay architecture tuned for FPGAs. DaCO pushes the
        performance boundaries of existing designs by exploiting static
        criticality information to support out-of-order execution inside each
        processing element. When compared to in-order designs, DaCO delivers up
        to 2.4$\times$ improvement in performance.
    }
	\cvitem{2009 -- 2012}{Undergraduate}{Imperial College London}{
        BEng in Electrical \& Electronics Engineering, graduated with a second-upper class honors degree.
    }
\end{cvtable}


% \cvsubsection{Study}
% \begin{cvtable}[1.5]
% 	\cvitem{2006 -- 2008}{Master Studies Panda Science}{Panda Academy}
% 		{Focus: Advanced rice hat studies and nouveau rain-reflecting cover
% 		materials.}
% 	\cvitem{}{Master Theses ($\varnothing\,	1,0$)}{Asian Rice Hat Institute}
% 		{Impact on solar radiation onto rice hat cover materials with special
% 		attention to water resistance.}
% 	\cvitem{2003 -- 2006}{Bachelor Studies PandaScience}{Panda Academy}
% 		{Focus: Bamboo morphology and its usage in different craftmanships.}
% 	\cvitem{}{Bachelor Theses ($\varnothing\,	1,0$)}{Bamboo Institute}
% 		{The bambo flute: An underestimated instrument in orchestras?}
% \end{cvtable}


\cvsection{Notable Publications\hfill {\small\href{https://scholar.google.com/citations?user=tRI56rsAAAAJ}{\faGraduationCap\hspace{0.05in}Google Scholar}}}
\begin{cvtable}
    \cvpubitem{DaCO: A High-Performance Token Dataflow Coprocessor Overlay for FPGAs}{Siddhartha, Nachiket Kapre}
		{International Conference on Field-Programmable Technology}{2018}
    \cvpubitem{Hoplite-Q: Priority-Aware Routing in FPGA Overlay NoCs}{Siddhartha, Nachiket Kapre}{IEEE 26th Annual International Symposium on Field-Programmable Custom Computing Machines}{2018}
    \cvpubitem{LUXOR: An FPGA Logic Cell Architecture for Eficient Compressor Tree Implementations}{Seyedramin Rasoulinezhad, Siddhartha, Hao Zhou, Lingli Wang, David Boland, Philip Leong}{ACM/SIGDA International Symposium on FPGAs}{2020}
    \cvpubitem{Long Short-Term Memory for Radio Frequency Spectral Prediction and its Real-Time FPGA Implementation}{Siddhartha, Yee Hui Lee, Duncan Moss, Julian Faraone, Perry Blackmore, Daniel Salmond, David Boland, and Philip Leong}{IEEE Military Communications Conference (MILCOM)}{2018}
    \cvpubitem{Real-Time Automatic Modulation Classification using RFSoC}{Stephen Tridgell, David Boland, Philip Leong, Ryan Kastner, Alireza Khodamoradi, Siddhartha}{27th Reconfigurable Architectures Workshop}{2020}
\end{cvtable}

% \cvsection{Awards}
% \begin{cvtable}
%     \cvitem{2013}{Richard Newton Young Fellow Award}{Design Automation Conference}{}
% \end{cvtable}


% \cvsection{Extra-Curricular Activities}
% \begin{cvtable}
% 	\cvitemshort{Relaxing}{Master the fine art of relaxing everywhere}
% 	\cvitemshort{Music}{Playing the bamboo flute in the 1st Panda Orchestra}
% 	\cvitemshort{Education}{Teaching young pandas to be more panda-like}
% \end{cvtable}


% \newpage
% \makebacksidebar
% 
% 
% \cvsection{section}
% \cvsubsection{Subsection}
% \begin{cvtable}
% 	\cvitem{<dates>}{<cv-item title>}{<location>}{<optional: description>}
% \end{cvtable}
% 
% \cvsection{cvitem}
% \cvsubsection{Multi-line with longer description}
% \begin{cvtable}
% 	\cvitem{date}{Description}{location}{Some longer and more detailed 
% 		description, that takes two lines	of space instead of only one.}
% 	\cvitem{date}{Description}{location}{Some longer and more detailed 
% 		description, that takes two lines	of space instead of only one.}
% 	\cvitem{date}{Description}{location}{Some longer and more detailed 
% 		description, that takes two lines	of space instead of only one.}
% \end{cvtable}
% 
% \cvsubsection{One-line without description}
% \begin{cvtable}
% 	\cvitem{Award}{One-line description}{Sponsor}{}
% 	\cvitem{Award}{One-line description}{Sponsor}{}
% 	\cvitem{Award}{One-line description}{Sponsor}{}
% \end{cvtable}
% 
% \cvsection{cvitemshort}
% \cvsubsection{One-line}
% \begin{cvtable}
% 	\cvitemshort{Key}{Some further description}
% 	\cvitemshort{Key}{Some further description}
% 	\cvitemshort{Key}{Some further description}
% \end{cvtable}
% 
% \cvsubsection{Multi-line with longer description}
% \begin{cvtable}
% 	\cvitemshort{Key}{Some further description. Can fill even more than
% 		only one single line while still keeping the correct indendation level.}
% 	\cvitemshort{Key}{Some further description. Can fill even more than
% 		only one single line while still keeping the correct indendation level.}
% 	\cvitemshort{Key}{Some further description. Can fill even more than
% 		only one single line while still keeping the correct indendation level.}
% \end{cvtable}
% 
% \cvsection{cvpubitem}
% \begin{cvtable}
% 	\cvpubitem{Publication title}{Authors}{Journal}{Year}
% 	\cvpubitem{Publication title}{Authors}{Journal}{Year}
% 	\cvpubitem{Publication title that is spanning over multiple lines and still
% 		does not look too bad}{Authors}{Journal}{Year}
% \end{cvtable}

\vspace{-0.1in}
\cvsignature

\end{document} 
