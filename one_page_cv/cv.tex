% FortySecondsCV LaTeX template
% Copyright © 2019 René Wirnata <rene.wirnata@pandascience.net>
% Licensed under the 3-Clause BSD License. See LICENSE file for details.
%
% Attributions
% ------------
% * fortysecondscv is based on the twentysecondcv class by Carmine Spagnuolo 
%   (cspagnuolo@unisa.it), released under the MIT license and available under
%   https://github.com/spagnuolocarmine/TwentySecondsCurriculumVitae-LaTex
% * further attributions are indicated immediately before corresponding code


%-------------------------------------------------------------------------------
%                             ADDITIONAL PACKAGES
%-------------------------------------------------------------------------------
\documentclass[
  a4paper, 
%   showframes,
%   vline=2.2em,
%   maincolor=cvgreen,
%   sectioncolor=red,
%   subsectioncolor=orange,
%   itemtextcolor=black!80,
%   sidebarwidth=0.4\paperwidth,
%   topbottommargin=0.03\paperheight,
%   leftrightmargin=20pt,
%   proilepicsize=4.5cm,
]{fortysecondscv}

\definecolor{linkcolor}{HTML}{005A87}

% improve word spacing and hyphenation
\usepackage{microtype}
\usepackage{ragged2e}

% take care of proper font encoding
\ifxetexorluatex
	\usepackage{fontspec}
	\defaultfontfeatures{Ligatures=TeX}
% \newfontfamily\headingfont[Path = fonts/]{segoeuib.ttf} % local font
\else
	\usepackage[utf8]{inputenc}
	\usepackage[T1]{fontenc}
% \usepackage[sfdefault]{noto} % use noto google font
\fi

% enable mathematical syntax for some symbols like \varnothing
\usepackage{amssymb}

% bubble diagram configuration
\usepackage{smartdiagram}
\smartdiagramset{
  % defaut font size is \large, so adjust to harmonize with sidebar layout
  bubble center node font = \footnotesize,
  bubble node font = \footnotesize,
  % default: 4cm/2.5cm; make minimum diameter relative to sidebar size
  bubble center node size = 0.4\sidebartextwidth,
  bubble node size = 0.25\sidebartextwidth,
  distance center/other bubbles = 1.5em,
  % set center bubble color
  bubble center node color = maincolor!70,
  % define the list of colors usable in the diagram
  set color list = {maincolor!10, maincolor!40,
  maincolor!20, maincolor!60, maincolor!35},
  % sets the opacity at which the bubbles are shown
  bubble fill opacity = 0.8,
}


%-------------------------------------------------------------------------------
%                            PERSONAL INFORMATION
%-------------------------------------------------------------------------------
%% mandatory information
% your name
\cvname{Siddhartha}
% job title/career
\cvjobtitle{Software/Hardware Engineer}

%% optional information
% profile picture
\cvprofilepic{pics/profile_compressed.jpg}

% short address/location, use \newline if more than 1 line is required
% \cvaddress{Singapore}
% phone number
% \cvphone{+61 411 77 1811}
% email address
\cvmail{sidmontu@gmail.com}
% personal website
\cvsite{https://sidmontu.github.io}
% github
\cvgithub{https://github.com/sidmontu/}{\small github.com/sidmontu}
% linkedin 
\cvlinkedin{https://www.linkedin.com/in/siddhartha-63249064/}{\small linkedin.com/in/siddhartha-63249064}
% google scholar 
% \cvscholar{https://scholar.google.com/citations?user=tRI56rsAAAAJ}{Google Scholar}

%-------------------------------------------------------------------------------
%                              SIDEBAR 1st PAGE
%-------------------------------------------------------------------------------
% add more profile sections to sidebar on first page
\addtofrontsidebar{
	% include gosquare national flags from https://github.com/gosquared/flags;
	% naming according to ISO 3166-1 alpha-2 country codes
	\graphicspath{{pics/flags/}}

    \profilesection{Personal Statement}
	\aboutme{
        % Computer engineer/researcher with a strong academic track record
        % looking for a fresh challenge. I am especially interested in hardware
        % design and/or machine-learning related projects, as I have relevant
        % experience in both areas. Lifelong learner open to picking up new
        % technology stacks when needed, and ready to work hard towards ambitious
        % targets.

        Experienced mid-career software engineer with a FPGA-focused research
        background looking for a fresh challenge. Especially interested in
        applied AI and ML engineering projects (e.g. MLOps), with/without
        hardware acceleration (e.g. FPGAs/ASICs) as an auxiliary goal.
	}

	% social network accounts incl. proper hyperlinks
% 	\profilesection{Social Network}
% 		\begin{icontable}{2.5em}{1em}
% 		    % overleaf still not supports Academicons and FontAwesome5 for XeLaTeX, which contain the overleaf logl...unbelievable...
% 			\social{\faGraduationCap}
% 				{https://scholar.google.com/citations?user=tRI56rsAAAAJ/}
% 				{Google Scholar}
% 			\social{\faGithub}
% 				{https://github.com/sidmontu/}
% 				{Github}
% 			\social{\faGlobe}
% 				{https://sidmontu.github.io/}
% 				{Personal Website}
% 		\end{icontable}

%	\profilesection{Languages}
%		\pointskill{\flag{CN.png}}{Chinese}{5}
%		\pointskill{\flag{DE.png}}{German}{3}
%  	\pointskill{\flag{GB.png}}{English}{3}
%  	\pointskill{\flag{FR.png}}{French}{3}

	\profilesection{Software Experience}
        \skill{\faLinux}{UNIX and CLI}
        \skill{\faPython}{Python3}
        \skill{\faCogs}{PyTorch / Tensorflow / spaCy}
        % \skill{\faToolbox}{MLOps tools such as MLFlow, Hub, and Label Studio}
        \skill{\faCuttlefish}{C/C++ programming}
        \skill{\faReact}{ReactJS / Redux / SASS}
        \skill{\faChartBar}{Data analysis/visualizations in R}
        % \skill{\faFont}{Typesetting with \LaTeX}
        % \skill{\faCodeBranch}{Version control, {\it e.g.} \texttt{git}}

	\profilesection{Hardware Experience}
        \skill{\faUniversity}{Foundational knowledge}
        \skill{\faKeyboard}{Verilog / Verilator}
        \skill{\faWrench}{FPGA toolchains: Vivado \& Quartus}
        \skill{\faTerminal}{VivadoHLS / PYNQ Frameworks}
        \skill{\faWifi}{RF communication (Ettus RFNoC Framework)}
        % \skill{\faLightbulb}{Learning SystemVerilog/UVM}

    \profilesection{Deployment Experience}
        \skill{\faAws}{AWS cloud services}
        \skill{\faDocker}{Docker \& docker-compose}
        \skill{\faCheck}{Unit-testing \& coverage}
        \skill{\faHammer}{TravisCI automated builds}
        % \skill{\faServer}{Serverless application framework}
        % \skill{\faBrain}{ML inference deployment and monitoring with cortex.dev}
}

%-------------------------------------------------------------------------------
%                              SIDEBAR 2nd PAGE
%-------------------------------------------------------------------------------
\addtobacksidebar{
	
	\profilesection{Diagrams}
	\chartlabel{Bubble Diagram}
	\begin{figure}\centering
		\smartdiagram[bubble diagram]{
			\textcolor{white}{\textbf{Being a}} \\ 
			\textcolor{white}{\textbf{Panda}}, % center bubble	
			\textcolor{black!90}{Eating},
			\textcolor{black!90}{Sleeping},
			\textcolor{black!90}{Rolling},
			\textcolor{black!90}{Playing},
			\textcolor{black!90}{Chilling}
		}
	\end{figure}

	\chartlabel{Wheel Chart}

	\wheelchart{4em}{2em}{%
  	20/3em/maincolor!50/Chill,
  	15/3em/maincolor!15/Play,
  	30/4em/maincolor!40/Sleep,
  	20/3em/maincolor!20/Eat
	}

	\profilesection{Barskills}
	\barskill{\faSkyatlas}{Wearing asian rice hats}{60}
	\barskill{\faImage}{Playing Chess}{30}
	\barskill{\faMusic}{Playing the bamboo flute}{50}

	\profilesection{Memberships}
	\begin{memberships}
		\membership[4em]{pics/logo.png}{PandaScience.net}
		\membership[4em]{pics/logo.png}{Some long text spanning over more than
			only one line}
		\membership[4em]{pics/logo.png}{\rule{\linewidth}{1pt}}
	\end{memberships}
}

\newcolumntype{M}[1]{>{\centering\arraybackslash}m{#1}}
\newcolumntype{L}[1]{>{\arraybackslash}m{#1}}
\newcolumntype{R}[1]{>{\raggedleft\arraybackslash}m{#1}}

%-------------------------------------------------------------------------------
%                         TABLE ENTRIES RIGHT COLUMN
%-------------------------------------------------------------------------------
\begin{document}

\makefrontsidebar

\cvsection{Work Experience}
\vspace{-0.2in}
\begin{cvtable}[3]
    \cvitem{Jun'21 -- Present}{Machine Learning Engineer (Contract)}{SAP Asia}{
        \renewcommand{\arraystretch}{1.2}% Tighter
        \begin{tabular}{L{4mm}L{100mm}}
            &\\[-0.1in]
            \raisebox{2\height}{\tikz \fill [cvblue] (-0.07,-0.2) rectangle (0.07,-0.05);} & {\small Part of the machine learning engineering team that builds scalable and production-ready ML solutions for internal stakeholders.}\\
            \raisebox{0.25\height}{\tikz \fill [cvblue] (-0.07,-0.2) rectangle (0.07,-0.05);} & {\small Worked with [Elastic|Open]Search, Docker, FastAPI, Agile, ReactJS, etc.}\\
        \end{tabular}
    }

    \cvitem{Feb'20 -- Sep'21}{Founder \& CTO}{inPact AI}{
        \renewcommand{\arraystretch}{1.2}% Tighter
        \begin{tabular}{L{4mm}L{100mm}}
            &\\[-0.1in]
            \raisebox{2\height}{\tikz \fill [cvblue] (-0.07,-0.2) rectangle (0.07,-0.05);} & {\small Led product development across all aspects of machine learning, frontend, backend, and deployment.}\\
            \raisebox{2\height}{\tikz \fill [cvblue] (-0.07,-0.2) rectangle (0.07,-0.05);} & {\small Hired and managed two contractors and an intern to meet regular product development targets.}\\ % Enforced good development practices such as version control, linters, code reviews, etc.}\\
            % \raisebox{4.75\height}{\tikz \fill [cvblue] (-0.07,-0.2) rectangle (0.07,-0.05);} & {\small Gained hands-on development experience with MLOps tools: Label studio for data annotation, MLFlow for model registry, Hub for dataset/feature store, and Cortex.dev for model deployment.}\\
            % \raisebox{4.75\height}{\tikz \fill [cvblue] (-0.07,-0.2) rectangle (0.07,-0.05);} & {\small Hands-on experience with ReactJS and serverless frameworks. Setup inPact’s cloud infrastructure on AWS -- used a myriad of AWS cloud services such as Cognito, DynamoDB, S3, SNS, Lambda, etc.}\\
            \raisebox{2\height}{\tikz \fill [cvblue] (-0.07,-0.2) rectangle (0.07,-0.05);} & {\small Hands-on product development experience with ReactJS, serverless framework, and AWS cloud infrastructure.}\\
            \raisebox{2\height}{\tikz \fill [cvblue] (-0.07,-0.2) rectangle (0.07,-0.05);} & {\small Handled business functions such as customer acquisition, corporate partnerships, investor relations, fundraising, marketing, and more.}\\
            \raisebox{4.75\height}{\tikz \fill [cvblue] (-0.07,-0.2) rectangle (0.07,-0.05);} & {\small Raised S\$75K pre-seed from Entrepreneur First, an international VC firm funded by Reid Hoffman (founder of LinkedIn), founders of DeepMind and PayPal, and some of the top investors in the world.}\\
        \end{tabular}
    }

	\cvitem{Oct'17 -- Dec'19}{Postdoctoral Research Associate}{University of Sydney}{
        \renewcommand{\arraystretch}{1.2}% Tighter
        \begin{tabular}{L{4mm}L{100mm}}
            &\\[-0.1in]
            \raisebox{4.75\height}{\tikz \fill [cvblue] (-0.07,-0.2) rectangle (0.07,-0.05);} & {\small Conducted research on: next-generation FPGA (overlay) architectures, low-precision deep neural networks, on-chip machine learning, and RF communication systems (Ettus RFNoC framework).}\\
            \raisebox{2\height}{\tikz \fill [cvblue] (-0.07,-0.2) rectangle (0.07,-0.05);} & {\small Core team member on a high-speed machine learning project for RF communications using FPGAs.}\\
            \raisebox{0.25\height}{\tikz \fill [cvblue] (-0.07,-0.2) rectangle (0.07,-0.05);} & {\small (Co-)Authored 5 peer-reviewed research papers/posters.}\\
            \raisebox{2\height}{\tikz \fill [cvblue] (-0.07,-0.2) rectangle (0.07,-0.05);} & {\small Supervised final year undergraduate projects, assisted with teaching/invigilation, and undertook sysadmin duties over lab resources.}\\
        \end{tabular}
    }

    % \cvitemm{Oct'17 -- Dec'19}{Postdoctoral Research Associate}{University of Sydney}{abcd}

% 	\cvitem{Jun -- Oct 2012}{Undergraduate Research Assistant}{Imperial College London}{
%         Under the University Research Opportunities Program (UROP), I embarked
%         on a summer project under the supervision of
%         \href{https://nachiket.github.io}{Nachiket Kapre}, which eventually
%         served as a foundation to my PhD research. The work produced during
%         this stint was published as a short paper in the 2014 IEEE
%         Field-Programmable Custom Computing Machines conference proceedings.
%     }
\end{cvtable}

\vspace{-0.1in}
\cvsection{Education}
\vspace{-0.1in}
\begin{cvtable}[1.5]
	\cvitem{2013 -- 2019}{Doctor of Philosophy}{Nanyang Technological University, Singapore}{
        \textit{\small Dissertation: Dataflow Optimized Overlays for FPGAs} \\
        \textit{\small Supervisor: \href{https://nachiket.github.io}{\textcolor{cvblue}{Dr. Nachiket Kapre}}} \\
        % This thesis introduces Dataflow Coprocessor Overlay (DaCO), a token
        % dataflow overlay architecture tuned for FPGAs. DaCO pushes the
        % performance boundaries of existing designs by exploiting static
        % criticality information to support out-of-order execution inside each
        % processing element. When compared to in-order designs, DaCO delivers up
        % to 2.4$\times$ improvement in performance.
        \renewcommand{\arraystretch}{1.2}% Tighter
        \begin{tabular}{L{4mm}L{100mm}}
        \raisebox{0\height}{\tikz \fill [cvblue] (-0.07,-0.2) rectangle (0.07,-0.05);} & {\footnotesize
            Developed DaCO, a Dataflow Coprocessor Overlay optimized for Arria
            10 FPGAs.
        }\\
        \raisebox{7.25\height}{\tikz \fill [cvblue] (-0.07,-0.2) rectangle (0.07,-0.05);} & {\footnotesize
            Key research contributions: (1) custom dataflow scheduling circuit
            that enables large-scale out-of-order instruction execution at
            runtime, (2) priority-aware NoC packet routing for
            criticality-aware dataflow communication, and (3) compiler support
            to optimize dataflow graphs for better runtime performance.
        }\\
        \raisebox{0\height}{\tikz \fill [cvblue] (-0.07,-0.2) rectangle (0.07,-0.05);} & {\footnotesize (Co-)Authored a total of 15 research papers/posters during the candidacy.}\\
        \end{tabular}
    }
    \cvitem{2009 -- 2012}{Bachelors of Engineering (BEng)}{Imperial College London}{
        {\small Faculty of Electrical \& Electronics Engineering, graduated with a second-upper class honors degree.}
    }
\end{cvtable}


% \cvsubsection{Study}
% \begin{cvtable}[1.5]
% 	\cvitem{2006 -- 2008}{Master Studies Panda Science}{Panda Academy}
% 		{Focus: Advanced rice hat studies and nouveau rain-reflecting cover
% 		materials.}
% 	\cvitem{}{Master Theses ($\varnothing\,	1,0$)}{Asian Rice Hat Institute}
% 		{Impact on solar radiation onto rice hat cover materials with special
% 		attention to water resistance.}
% 	\cvitem{2003 -- 2006}{Bachelor Studies PandaScience}{Panda Academy}
% 		{Focus: Bamboo morphology and its usage in different craftmanships.}
% 	\cvitem{}{Bachelor Theses ($\varnothing\,	1,0$)}{Bamboo Institute}
% 		{The bambo flute: An underestimated instrument in orchestras?}
% \end{cvtable}


\vspace{-0.1in}
\cvsection{Certifications}

% \hfill {\small\href{https://scholar.google.com/citations?user=tRI56rsAAAAJ}{\faGraduationCap\hspace{0.05in}Google Scholar}}}

\begin{cvtable}
    \cvpubitem{\textcolor{cvblue}{\href{https://www.coursera.org/specializations/machine-learning-engineering-for-production-mlops}{\it Machine Learning Engineering for Production (MLOps) Specialization}}}{Coursera MOOC offered by DeepLearning.AI}{4-module specialization focusing on best in-industry practices for deploying data-centric machine learning systems. Instructors include Andrew Ng (founder DeepLearning.AI and Coursera), Robert Crowe (Tensorflow Developer Engineer, Google), Laurence Moroney (Lead AI Advocate, Google), and more.}{Ongoing}
    \cvpubitem{\textcolor{cvblue}{\href{https://www.coursera.org/specializations/jhu-data-science}{\it Data Science Specialization}}}{Coursera MOOC offered by John Hopkins University}{10-module specialization that covers concepts and tools essential for building effective data science pipelines. Instructors include Jeff Leek, Roger D. Peng, and Brian Caffo, all of whom are professors at John Hopkins university.}{Apr 2016}
\end{cvtable}

% \begin{cvtable}
%     \cvpubitem{DaCO: A High-Performance Token Dataflow Coprocessor Overlay for FPGAs}{Siddhartha, Nachiket Kapre}
% 		{International Conference on Field-Programmable Technology}{2018}
%     % \cvpubitem{Hoplite-Q: Priority-Aware Routing in FPGA Overlay NoCs}{Siddhartha, Nachiket Kapre}{IEEE 26th Annual International Symposium on Field-Programmable Custom Computing Machines}{2018}
%     \cvpubitem{LUXOR: An FPGA Logic Cell Architecture for Eficient Compressor Tree Implementations}{Seyedramin Rasoulinezhad, Siddhartha, Hao Zhou, Lingli Wang, David Boland, Philip Leong}{ACM/SIGDA International Symposium on FPGAs}{2020}
%     \cvpubitem{Long Short-Term Memory for Radio Frequency Spectral Prediction and its Real-Time FPGA Implementation}{Siddhartha, Yee Hui Lee, Duncan Moss, Julian Faraone, Perry Blackmore, Daniel Salmond, David Boland, and Philip Leong}{IEEE Military Communications Conference (MILCOM)}{2018}
%     % \cvpubitem{Real-Time Automatic Modulation Classification using RFSoC}{Stephen Tridgell, David Boland, Philip Leong, Ryan Kastner, Alireza Khodamoradi, Siddhartha}{27th Reconfigurable Architectures Workshop}{2020}
% \end{cvtable}

% \cvsection{Awards}
% \begin{cvtable}
%     \cvitem{2013}{Richard Newton Young Fellow Award}{Design Automation Conference}{}
% \end{cvtable}


% \cvsection{Extra-Curricular Activities}
% \begin{cvtable}
% 	\cvitemshort{Relaxing}{Master the fine art of relaxing everywhere}
% 	\cvitemshort{Music}{Playing the bamboo flute in the 1st Panda Orchestra}
% 	\cvitemshort{Education}{Teaching young pandas to be more panda-like}
% \end{cvtable}


% \newpage
% \makebacksidebar
% 
% 
% \cvsection{section}
% \cvsubsection{Subsection}
% \begin{cvtable}
% 	\cvitem{<dates>}{<cv-item title>}{<location>}{<optional: description>}
% \end{cvtable}
% 
% \cvsection{cvitem}
% \cvsubsection{Multi-line with longer description}
% \begin{cvtable}
% 	\cvitem{date}{Description}{location}{Some longer and more detailed 
% 		description, that takes two lines	of space instead of only one.}
% 	\cvitem{date}{Description}{location}{Some longer and more detailed 
% 		description, that takes two lines	of space instead of only one.}
% 	\cvitem{date}{Description}{location}{Some longer and more detailed 
% 		description, that takes two lines	of space instead of only one.}
% \end{cvtable}
% 
% \cvsubsection{One-line without description}
% \begin{cvtable}
% 	\cvitem{Award}{One-line description}{Sponsor}{}
% 	\cvitem{Award}{One-line description}{Sponsor}{}
% 	\cvitem{Award}{One-line description}{Sponsor}{}
% \end{cvtable}
% 
% \cvsection{cvitemshort}
% \cvsubsection{One-line}
% \begin{cvtable}
% 	\cvitemshort{Key}{Some further description}
% 	\cvitemshort{Key}{Some further description}
% 	\cvitemshort{Key}{Some further description}
% \end{cvtable}
% 
% \cvsubsection{Multi-line with longer description}
% \begin{cvtable}
% 	\cvitemshort{Key}{Some further description. Can fill even more than
% 		only one single line while still keeping the correct indendation level.}
% 	\cvitemshort{Key}{Some further description. Can fill even more than
% 		only one single line while still keeping the correct indendation level.}
% 	\cvitemshort{Key}{Some further description. Can fill even more than
% 		only one single line while still keeping the correct indendation level.}
% \end{cvtable}
% 
% \cvsection{cvpubitem}
% \begin{cvtable}
% 	\cvpubitem{Publication title}{Authors}{Journal}{Year}
% 	\cvpubitem{Publication title}{Authors}{Journal}{Year}
% 	\cvpubitem{Publication title that is spanning over multiple lines and still
% 		does not look too bad}{Authors}{Journal}{Year}
% \end{cvtable}

\vspace{-0.1in}
\cvsignature

\end{document} 
