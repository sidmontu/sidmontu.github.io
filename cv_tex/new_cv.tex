%%%%%%%%%%%%%%%%%%%%%%%%%%%%%%%%%%%%%%%%%%%%%%%%%%%%%%%%%%%%%%%%%%%%%%%%
%%%%%%%%%%%%%%%%%%%%%% Simple LaTeX CV Template %%%%%%%%%%%%%%%%%%%%%%%%
%%%%%%%%%%%%%%%%%%%%%%%%%%%%%%%%%%%%%%%%%%%%%%%%%%%%%%%%%%%%%%%%%%%%%%%%

%%%%%%%%%%%%%%%%%%%%%%%%%%%%%%%%%%%%%%%%%%%%%%%%%%%%%%%%%%%%%%%%%%%%%%%%
%% NOTE: If you find that it says                                     %%
%%                                                                    %%
%%                           1 of ??                                  %%
%%                                                                    %%
%% at the bottom of your first page, this means that the AUX file     %%
%% was not available when you ran LaTeX on this source. Simply RERUN  %%
%% LaTeX to get the ``??'' replaced with the number of the last page  %%
%% of the document. The AUX file will be generated on the first run   %%
%% of LaTeX and used on the second run to fill in all of the          %%
%% references.                                                        %%
%%%%%%%%%%%%%%%%%%%%%%%%%%%%%%%%%%%%%%%%%%%%%%%%%%%%%%%%%%%%%%%%%%%%%%%%

%%%%%%%%%%%%%%%%%%%%%%%%%%%% Document Setup %%%%%%%%%%%%%%%%%%%%%%%%%%%%

% Don't like 10pt? Try 11pt or 12pt
\documentclass[10pt]{article}

% The automated optical recognition software used to digitize resume
% information works best with fonts that do not have serifs. This
% command uses a sans serif font throughout. Uncomment both lines (or at
% least the second) to restore a Roman font (i.e., a font with serifs).
%\usepackage{times}
%\renewcommand{\familydefault}{\sfdefault}

% This is a helpful package that puts math inside length specifications
\usepackage{calc}
\usepackage{comment}
\usepackage{graphicx}

% Simpler bibsection for CV sections
% (thanks to natbib for inspiration)
\makeatletter
\newlength{\bibhang}
\setlength{\bibhang}{1em} %1em}
\newlength{\bibsep}
 {\@listi \global\bibsep\itemsep \global\advance\bibsep by\parsep}
\newenvironment{bibsection}%
        {\begin{enumerate}{}{%
%        {\begin{list}{}{%
       \setlength{\leftmargin}{\bibhang}%
       \setlength{\itemindent}{-\leftmargin}%
       \setlength{\itemsep}{\bibsep}%
       \setlength{\parsep}{\z@}%
        \setlength{\partopsep}{0pt}%
        \setlength{\topsep}{0pt}}}
        {\end{enumerate}\vspace{-.6\baselineskip}}
%        {\end{list}\vspace{-.6\baselineskip}}
\makeatother

% Layout: Puts the section titles on left side of page
\reversemarginpar

%
%         PAPER SIZE, PAGE NUMBER, AND DOCUMENT LAYOUT NOTES:
%
% The next \usepackage line changes the layout for CV style section
% headings as marginal notes. It also sets up the paper size as either
% letter or A4. By default, letter was used. If A4 paper is desired,
% comment out the letterpaper lines and uncomment the a4paper lines.
%
% As you can see, the margin widths and section title widths can be
% easily adjusted.
%
% ALSO: Notice that the includefoot option can be commented OUT in order
% to put the PAGE NUMBER *IN* the bottom margin. This will make the
% effective text area larger.
%
% IF YOU WISH TO REMOVE THE ``of LASTPAGE'' next to each page number,
% see the note about the +LP and -LP lines below. Comment out the +LP
% and uncomment the -LP.
%
% IF YOU WISH TO REMOVE PAGE NUMBERS, be sure that the includefoot line
% is uncommented and ALSO uncomment the \pagestyle{empty} a few lines
% below.
%

%% Use these lines for letter-sized paper
\usepackage[paper=letterpaper,
            %includefoot, % Uncomment to put page number above margin
            marginparwidth=1.2in,     % Length of section titles
            marginparsep=.05in,       % Space between titles and text
            margin=1in,               % 1 inch margins
            includemp]{geometry}

%% Use these lines for A4-sized paper
%\usepackage[paper=a4paper,
%            %includefoot, % Uncomment to put page number above margin
%            marginparwidth=30.5mm,    % Length of section titles
%            marginparsep=1.5mm,       % Space between titles and text
%            margin=25mm,              % 25mm margins
%            includemp]{geometry}

%% More layout: Get rid of indenting throughout entire document
\setlength{\parindent}{0in}

\usepackage[shortlabels]{enumitem}

%% Reference the last page in the page number
%
% NOTE: comment the +LP line and uncomment the -LP line to have page
%       numbers without the ``of ##'' last page reference)
%
% NOTE: uncomment the \pagestyle{empty} line to get rid of all page
%       numbers (make sure includefoot is commented out above)
%
\usepackage{fancyhdr,lastpage}
\pagestyle{fancy}
%\pagestyle{empty}      % Uncomment this to get rid of page numbers
\fancyhf{}\renewcommand{\headrulewidth}{0pt}
\fancyfootoffset{\marginparsep+\marginparwidth}
\newlength{\footpageshift}
\setlength{\footpageshift}
          {0.5\textwidth+0.5\marginparsep+0.5\marginparwidth-2in}
\lfoot{\hspace{\footpageshift}%
       \parbox{4in}{\, \hfill %
                    \arabic{page} of \protect\pageref*{LastPage} % +LP
%                    \arabic{page}                               % -LP
                    \hfill \,}}

% Finally, give us PDF bookmarks
\usepackage{color,hyperref}
\definecolor{darkblue}{rgb}{0.0,0.0,0.3}
\hypersetup{colorlinks,breaklinks,
            linkcolor=darkblue,urlcolor=darkblue,
            anchorcolor=darkblue,citecolor=darkblue}

%%%%%%%%%%%%%%%%%%%%%%%% End Document Setup %%%%%%%%%%%%%%%%%%%%%%%%%%%%


%%%%%%%%%%%%%%%%%%%%%%%%%%% Helper Commands %%%%%%%%%%%%%%%%%%%%%%%%%%%%

% The title (name) with a horizontal rule under it
% (optional argument typesets an object right-justified across from name
%  as well)
%
% Usage: \makeheading{name}
%        OR
%        \makeheading[right_object]{name}
%
% Place at top of document. It should be the first thing.
% If ``right_object'' is provided in the square-braced optional
% argument, it will be right justified on the same line as ``name'' at
% the top of the CV. For example:
%
%       \makeheading[\emph{Curriculum vitae}]{Your Name}
%
% will put an emphasized ``Curriculum vitae'' at the top of the document
% as a title. Likewise, a picture could be included:
%
%   \makeheading[\includegraphics[height=1.5in]{my_picutre}]{Your Name}
%
% the picture will be flush right across from the name.
\newcommand{\makeheading}[2][]%
        {\hspace*{-\marginparsep minus \marginparwidth}%
         \begin{minipage}[t]{\textwidth+\marginparwidth+\marginparsep}%
             {\large \bfseries #2 \hfill #1}\\[-0.15\baselineskip]%
                 \rule{\columnwidth}{1pt}%
         \end{minipage}}

% The section headings
%
% Usage: \section{section name}
\renewcommand{\section}[1]{\pagebreak[3]%
    \hyphenpenalty=10000%
    \vspace{1.3\baselineskip}%
    \phantomsection\addcontentsline{toc}{section}{#1}%
    \noindent\llap{\scshape\smash{\parbox[t]{\marginparwidth}{\raggedright #1}}}%
    \vspace{-\baselineskip}\par}

% An itemize-style list with lots of space between items
\newenvironment{outerlist}[1][\enskip\textbullet]%
        {\begin{itemize}[#1,leftmargin=*]}{\end{itemize}%
         \vspace{-.6\baselineskip}}

% An environment IDENTICAL to outerlist that has better pre-list spacing
% when used as the first thing in a \section
\newenvironment{lonelist}[1][\enskip\textbullet]%
        {\begin{list}{#1}{%
        \setlength{\partopsep}{0pt}%
        \setlength{\topsep}{0pt}}}
        {\end{list}\vspace{-.6\baselineskip}}

% An itemize-style list with little space between items
\newenvironment{innerlist}[1][\enskip\textbullet]%
        {\begin{itemize}[#1,leftmargin=*,parsep=0pt,itemsep=0pt,topsep=0pt,partopsep=0pt]}
        {\end{itemize}}

% An environment IDENTICAL to innerlist that has better pre-list spacing
% when used as the first thing in a \section
\newenvironment{loneinnerlist}[1][\enskip\textbullet]%
        {\begin{itemize}[#1,leftmargin=*,parsep=0pt,itemsep=0pt,topsep=0pt,partopsep=0pt]}
        {\end{itemize}\vspace{-.6\baselineskip}}

% To add some paragraph space between lines.
% This also tells LaTeX to preferably break a page on one of these gaps
% if there is a needed pagebreak nearby.
\newcommand{\blankline}{\quad\pagebreak[3]}
\newcommand{\halfblankline}{\quad\vspace{-0.5\baselineskip}\pagebreak[3]}

% Uses hyperref to link DOI
\newcommand\doilink[1]{\href{http://dx.doi.org/#1}{#1}}
\newcommand\doi[1]{doi:\doilink{#1}}

% For \url{SOME_URL}, links SOME_URL to the url SOME_URL
\providecommand*\url[1]{\href{#1}{#1}}
% Same as above, but pretty-prints SOME_URL in teletype fixed-width font
\renewcommand*\url[1]{\href{#1}{\texttt{#1}}}

% For \email{ADDRESS}, links ADDRESS to the url mailto:ADDRESS
\providecommand*\email[1]{\href{mailto:#1}{#1}}
% Same as above, but pretty-prints ADDRESS in teletype fixed-width font
%\renewcommand*\email[1]{\href{mailto:#1}{\texttt{#1}}}

%\providecommand\BibTeX{{\rm B\kern-.05em{\sc i\kern-.025em b}\kern-.08em
%    T\kern-.1667em\lower.7ex\hbox{E}\kern-.125emX}}
%\providecommand\BibTeX{{\rm B\kern-.05em{\sc i\kern-.025em b}\kern-.08em
%    \TeX}}
\providecommand\BibTeX{{B\kern-.05em{\sc i\kern-.025em b}\kern-.08em
    \TeX}}
\providecommand\Matlab{\textsc{Matlab}}

%%%%%%%%%%%%%%%%%%%%%%%% End Helper Commands %%%%%%%%%%%%%%%%%%%%%%%%%%%

%%%%%%%%%%%%%%%%%%%%%%%%% Begin CV Document %%%%%%%%%%%%%%%%%%%%%%%%%%%%

\begin{document}
\makeheading{{\Huge Siddhartha}\hfill{\normalsize FPGAs / Machine Learning Engineer}}

\section{Contact Information}

% NOTE: Mind where the & separators and \\ breaks are in the following
%       table.
%
% ALSO: \rcollength is the width of the right column of the table
%       (adjust it to your liking; default is 1.85in).
%
\newlength{\rcollength}\setlength{\rcollength}{1.4in}%
%
\begin{tabular}[t]{@{}p{\textwidth-\rcollength-1.8in}p{\rcollength}}
%\href{http://www.cse.osu.edu/}%
%     {Department of Computer Science and Engineering} & \\
%\href{http://www.osu.edu/}{The Ohio State University}
\includegraphics[width=12px]{email_logo}\ \ \ \textcolor{black}{\textsf{sidmontu@gmail.com}} & 
	\includegraphics[width=12px]{telephone_logo}\ \ \ {\textsf{+65 9857-4171}} \\
    \includegraphics[width=12px]{globe_logo}\ \ \ \textcolor{black}{\href{https://sidmontu.github.io}{\textsf{https://sidmontu.github.io}}} & 
	\includegraphics[width=12px]{skype_logo}\ \ \ {\textsf{sidmontu}} \\
	\multicolumn{2}{@{}}{\includegraphics[width=12px]{home_logo}\ \ \ {\textsf{7, Woodsvale Condominium, \#06-13, Woodlands Drive 72, Singapore 738092}}} \\
\end{tabular}

%\section{Objective}

%Insert text here if you want to
%\begin{innerlist}
%\item More information and auxiliary documents can be found at\\\url{http://www.tedpavlic.com/facjobsearch/}
%\end{innerlist}

\section{Interests}

FPGAs, Machine Learning, Radio, Computer Architecture

\section{Professional Summary}
Seasoned researcher interested in FPGAs and efficient architectures for deep
learning applications. My recent projects involve developing high-speed,
low-latency machine learning solutions for RF communication systems on the
Xilinx RFSoC board, building token dataflow overlay architectures for
accelerating highly sparse and irregular workloads, and more. I am a strong
programmer and lifelong learner who has a keen interest in doing cutting-edge
research and development.

\section{Experience}

\textbf{Posdoctoral Research Associate} \hfill {October 2017 to December 2019}
\begin{innerlist}

\item[] Computering Engineering Lab,\\
        School of Electrical and Information Engineering\\
		University of Sydney\\
        Supervisors: \href{https://sydney.edu.au/engineering/people/philip.leong.php}{Philip Leong}, \href{https://sydney.edu.au/engineering/people/david.boland.php}{David Boland}
\end{innerlist}
\vspace{0.1in}
\textbf{Research Assistant} \hfill {July 2012 to September 2012}
\begin{innerlist}

\item[] Circuits and Systems Research Group,\\
        Electrical and Electronics Engineering Department\\
		Imperial College London\\
        Supervisor: \href{https://nachiket.github.io}{Nachiket Kapre}
\end{innerlist}

\section{Education}

\href{http://www.ntu.edu.sg}{\textbf{Nanyang Technological University (NTU)}},
Singapore
\begin{outerlist}

\item[] PhD,
        \href{http://scse.ntu.edu.sg/}
			{Computer Science \& Engineering},
             February 2019
        \begin{innerlist}
        \item Dissertation Title: \emph{Dataflow Optimized Overlays for FPGAs}
        \item Advisor:
              \href{https://nachiket.github.io/}{Nachiket Kapre}
        \end{innerlist}
\end{outerlist}
\vspace{.1in}
\href{http://www.imperial.ac.uk/}{\textbf{Imperial College London}},
London, United Kingdom 
\begin{outerlist}
\item[] BEng,
        \href{http://www.imperial.ac.uk/electrical-engineering}
			{Electrical \& Electronics Engineering}, June 2012 
\end{outerlist}

% Add a little space to nudge next ``Conference Publications'' marginpar
% down to make room for tall ``Submitted Journal Publications''
% marginpar. If there are enough submitted journal publications, this
% space will not be needed (and should be removed).
%\vspace{0.1in}

% \section{Expertise}
% {\bf Hardware Design / Hardware Acceleration}
% \begin{innerlist}
% \item {\bf Verilog} : Very competent with the language -- completed multiple large research projects using associated compilation/simulation tools.
% \item {\bf Vivado/Quartus} : Active contributor to the FPGA research field. Familiar with FPGA architectures, and comfortable with using Xilinx/Intel vendor tools.
% \item {\bf VivadoHLS} : Actively learning and using the environment to prototype and implement hardware designs in research projects.
% % \item {\bf SystemC/OpenCL} : Light introductory experience, familiar with fundamental concepts.
% % \item {\bf CUDA} : Familiar with CUDA programming framework -- used for projects in class assignments and research projects.
% \end{innerlist}
% \vspace{0.2in}
% {\bf Software Engineering}
% \begin{innerlist}
% \item {\bf C/C$++$} : Competent and very comfortable with the programming environment. Developed and managed multiple C/C$++$-based projects, and familiar with development tools and good design practices.
% %\item {\bf Java} : Managed and wrote extensions to an existing large Java-based project written for software-based hardware simulation for quick design space exploration. Competent with most common features of the programming language.
% \item {\bf Command line tools} : Strong preference for Unix-based environments -- comfortable working with popular command line tools (e.g. git, sed, awk, grep, etc).
% \item {\bf Python} : Primary choice of language for most software development projects lately.
% \item {\bf Tensorflow/Tensorpack} : Currently use extensively to design and train neural networks for various research projects.
% \item {\bf R} : Primary choice of programming environment for data analysis. Have completed a 10-module data science specialization course offered by the John Hopkins University on Coursera that was taught in R.
% \item {\bf \LaTeX} : Primary choice for typesetting any technical reports, or conference/journal publications. This CV was built on an existing open-source \LaTeX\ template.
% %\item {\bf HTML/CSS/Javascript} : Web design + mobile application development as a hobby. Attended a HTML5 web/app design bootcamp and participated in several local hackathons.
% \end{innerlist}

% \section{Awards}
% Richard Newton Young Fellow Award, Design Automation Conference \hfill June 2013\\

\section{Teaching}

{\bf Project Supervisor, University of Sydney} \hfill {Semesters 2018--2019}
\begin{innerlist}[rightmargin=\dimexpr\linewidth-6.5cm-\leftmargin\relax] 
\item[] Responsibilities: Guidance and supervision to undergraduate students on their final year projects.
\end{innerlist}
\vspace{0.1in}
{\bf Teaching Assistant, NTU} \hfill {Semesters 2014--2016}
\begin{innerlist}

\item[] CE4054 - Programmable System on Chip,\\
		CE4052 - Embedded Software Development\\
\end{innerlist}

\section{Extra-Curricular}
{\bf \href{https://www.coursera.org/account/accomplishments/specialization/YP6NMPR4X3M9}{Data Science Specialization}}\hfill{March 2015 -- April 2016}\\
{\it John Hopkins University}
\begin{innerlist}
\item Offered via the Coursera platform, the Data Science Specialization teaches how to use the tools of the trade, think analytically about complex problems, manage large data sets, employ statistical methodologies, create visualizations, build and evaluate machine learning algorithms, publish reproducible analyses, and develop data products.
\end{innerlist}
% \vspace{0.1in}
% {\bf Communication Coach}\hfill {January 2014 -- November 2016}\\
% {\it School of Humanities and Social Sciences, NTU}
% \begin{innerlist}
% \item A university-wide mentoring program to assist undergraduate/graduate students on written \& verbal communication skills.
% \end{innerlist}

\halfblankline

%\section{References}

%Nachiket Kapre
%\begin{innerlist}
%\item[] Assistant Professor,\hfill{\includegraphics[width=12px]{email_logo}\ \ \ nachiket@uwaterloo.ca}\\
%Electrical and Computer Engineering,\\
%University of Waterloo, Canada 
%\end{innerlist}

%\halfblankline

\newpage
\section{Peer-Reviewed Publications}
\textbf{\large Full Papers (Conferences/Journals)}
\vspace{0in}
\begin{enumerate}
    \item Seyedramin Rasoulinezhad, {\bf Siddhartha}, Hao Zhou, Lingli Wang, David Boland, and Philip Leongo ``LUXOR: An FPGA Logic Cell Architecture for Efficient Compressor Tree Implementations" \emph{Proceedings of the 2020 ACM/SIGDA International Symposium on Field-Programmable Gate Arrays (FPGA)}, February 2020\newline[\href{#}{\texttt{DOI: TBA}}]
    \item {\bf Siddhartha}, and Nachiket Kapre ``DaCO: A High-Performance Token Dataflow Coprocessor Overlay for FPGAs" \emph{International Conference on Field-Programmable Technology}, December 2018\newline[\href{https://doi.org/10.1109/FPT.2018.00032}{\texttt{DOI: 10.1109/FPT.2018.00032}}]
    \item {\bf Siddhartha}, Yee Hui Lee, Duncan Moss, Julian Faraone, Perry Blackmore, Daniel Salmond, David Boland, and Philip Leong ``Long Short-Term Memory for Radio Frequency Spectral Prediction and its Real-Time FPGA Implementation" \emph{IEEE Military Communications Conference (MILCOM)}, October 2018\newline[\href{https://doi.org/10.1109/MILCOM.2018.8599833}{\texttt{DOI: 10.1109/MILCOM.2018.8599833}}]
    \item {\bf Siddhartha}, Nachiket Kapre ``Hoplite-Q: Priority-Aware Routing in FPGA Overlay NoCs" \emph{IEEE 26th Annual International Symposium on Field-Programmable Custom Computing Machines}, May 2018\newline[\href{https://doi.org/10.1109/FCCM.2018.00012}{\texttt{DOI: 10.1109/FCCM.2018.00012}}]
    \item Gopalakrishna Hegde, {\bf Siddhartha}, Nachiket Kapre ``CaffePresso: Accelerating Convolutional Networks on Embedded SoCs'' \emph{ACM Transactions on Embedded Computing Systems (TECS)}, January 2018\newline[\href{https://doi.org/10.1145/3105925}{\texttt{DOI: 10.1145/3105925}}]
    \item {\bf Siddhartha}, Nachiket Kapre ``eBSP: Managing NoC traffic for BSP workloads on the 16-core Adapteva Epiphany-III Processor." \emph{Design, Automation, and Test in Europe}, March 2017\newline[\href{https://doi.org/10.23919/DATE.2017.7926961}{\texttt{DOI: 10.23919/DATE.2017.7926961}}]
    \item Gopalakrishna Hegde, \textbf{Siddhartha}, Nachiappan Ramasamy, Nachiket Kapre\newline``CaffePresso: An Optimized Library for Deep Learning on Embedded Accelerator-based platforms." \emph{International Conference on Compilers, Architecture, and Synthesis for Embedded Systems}, October 2016 (Best Paper Award)\newline[\href{https://doi.org/10.1145/2968455.2968511}{\texttt{DOI: 10.1145/2968455.2968511}}]
    \item Pradeep Moorthy, {\bf Siddhartha}, and Nachiket Kapre ``A Case for Embedded FPGA-based SoCs for Energy-Efficient Acceleration of Graph Problems."\newline\emph{Supercomputing Frontiers 2015}, March 2015\newline[\href{http://dx.doi.org/10.14529/jsfi150307}{\texttt{DOI: 10.14529/jsfi150307}}]
\end{enumerate}
\textbf{\large Short Papers / Posters / Workshops}
\begin{enumerate}
    \item Stephen Tridgell, David Boland, Philip Leong, and {\bf Siddhartha} ``Real-time Automatic Modulation Classification" \emph{International Conference on Field-Programmable Technology}, December 2019 (Poster)\newline[\texttt{DOI: TBA}]
    \item {\bf Siddhartha}, David Boland, Steve Wilton, Barry Flower, Perry Blackmore, and Philip Leong ``Simultaneous Inference and Training using On-FPGA Weight Perturbation Techniques" \emph{International Conference on Field-Programmable Technology}, December 2018 (Poster)\newline[\href{https://doi.org/10.1109/FPT.2018.00060}{\texttt{DOI: 10.1109/FPT.2018.00060}}]
    \item {\bf Siddhartha}, Nachiket Kapre ``Out-of-Order Dataflow Scheduling for FPGA Overlays." \emph{Overlay Architectures for FPGAs Workshop (co-located with FPGA 2017)}, February 2017 (Position Paper)\newline[\href{https://arxiv.org/abs/1704.08802}{\texttt{DOI: arXiv:1705.02734}}]
    \item Sidharth Maheshwari, Gourav Modi, {\bf Siddhartha}, Nachiket Kapre ``Vector FPGA Acceleration of 1-D DWT Computations using Sparse Matrix Skeletons." \emph{26th IEEE International Conference on Field-Programmable Logic and Applications}, August 2016 (Poster)\newline[\href{https://doi.org/10.1109/FPL.2016.7577361}{\texttt{DOI: 10.1109/FPL.2016.7577361}}]
    \item {\bf Siddhartha}, Nachiket Kapre ``Communication Optimization for the 16-core Epiphany Floating-Point Processor Array." \emph{24th IEEE International Symposium on Field-Programmable Custom Computing Machines}, May 2016 (Short Paper)\newline[\href{https://doi.org/10.1109/FCCM.2016.15}{\texttt{DOI: 10.1109/FCCM.2016.15}}]
    \item Gopalakrishna Hegde, {\bf Siddhartha}, Nachiappan Ramasamy, Vamsi Buddha, Nachiket Kapre ``Evaluating Embedded FPGA Accelerators for Deep Learning Applications." \emph{24th IEEE International Symposium on Field-Programmable Custom Computing Machines}, May 2016 (Short Paper)\newline[\href{https://doi.org/10.1109/FCCM.2016.14}{\texttt{DOI: 10.1109/FCCM.2016.14}}]
    \item Nachiket Kapre, Han Jianglei, Andrew Bean, Pradeep Moorthy, and {\bf Siddhartha} ``GraphMMU: Memory Management Unit for Sparse Graph Accelerators." \emph{22nd Reconfigurable Architectures Workshop (co-located with IPDPS)}, May 2015\newline[\href{https://doi.org/10.1109/IPDPSW.2015.101}{\texttt{DOI: 10.1109/IPDPSW.2015.101}}]
    \item {\bf Siddhartha}, Nachiket Kapre ``FPGA Acceleration of Irregular Iterative Computations using Criticality-Aware Dataflow Optimizations." \emph{International Symposium on Field-Programmable Gate Arrays}, February 2015 (Short Paper)\newline[\href{https://doi.org/10.1145/2684746.2689110}{\texttt{DOI: 10.1145/2684746.2689110}}]
    \item {\bf Siddhartha}, Nachiket Kapre ``Fanout Decomposition Dataflow Optimizations for FPGA-based Sparse LU Factorization." \emph{International Conference on Field-Programmable Technology}, December 2014 (Short Paper)\newline[\href{https://doi.org/10.1109/FPT.2014.7082787}{\texttt{DOI: 10.1109/FPT.2014.7082787}}]
    \item {\bf Siddhartha}, Nachiket Kapre ``Heterogeneous Dataflow Architectures for FPGA-based Sparse LU Factorization." \emph{The International Conference on Field Programmable Logic and Applications}, September 2014 (Short Paper)\newline[\href{https://doi.org/10.1109/FPL.2014.6927401}{\texttt{DOI: 10.1109/FPL.2014.6927401}}]
    \item Nachiket Kapre, {\bf Siddhartha} ``Limits of Statically-Scheduled Token Dataflow Processing." \emph{International workshop on Data-Flow Models (DFM) for Extreme Scale Computing (co-located with PACT 2014)}, August 2014\newline[\href{https://doi.org/10.1109/DFM.2014.21}{\texttt{DOI: 10.1109/DFM.2014.21}}]
    \item {\bf Siddhartha}, Nachiket Kapre ``Breaking Sequential Dependencies in FPGA-based Sparse LU Factorization." \emph{International Symposium on Field Programmable Custom Computing Machines}, May 2014 (Short Paper)\newline[\href{https://doi.org/10.1109/FCCM.2014.26}{\texttt{DOI: 10.1109/FCCM.2014.26}}]
\end{enumerate}


\end{document}

%%%%%%%%%%%%%%%%%%%%%%%%%% End CV Document %%%%%%%%%%%%%%%%%%%%%%%%%%%%%

%----------------------------------------------------------------------%
% The following is copyright and licensing information for
% redistribution of this LaTeX source code; it also includes a liability
% statement. If this source code is not being redistributed to others,
% it may be omitted. It has no effect on the function of the above code.
%----------------------------------------------------------------------%
% Copyright (c) 2007, 2008, 2009, 2010, 2011 by Theodore P. Pavlic
%
% Unless otherwise expressly stated, this work is licensed under the
% Creative Commons Attribution-Noncommercial 3.0 United States License. To
% view a copy of this license, visit
% http://creativecommons.org/licenses/by-nc/3.0/us/ or send a letter to
% Creative Commons, 171 Second Street, Suite 300, San Francisco,
% California, 94105, USA.
%
% THE SOFTWARE IS PROVIDED "AS IS", WITHOUT WARRANTY OF ANY KIND, EXPRESS
% OR IMPLIED, INCLUDING BUT NOT LIMITED TO THE WARRANTIES OF
% MERCHANTABILITY, FITNESS FOR A PARTICULAR PURPOSE AND NONINFRINGEMENT.
% IN NO EVENT SHALL THE AUTHORS OR COPYRIGHT HOLDERS BE LIABLE FOR ANY
% CLAIM, DAMAGES OR OTHER LIABILITY, WHETHER IN AN ACTION OF CONTRACT,
% TORT OR OTHERWISE, ARISING FROM, OUT OF OR IN CONNECTION WITH THE
% SOFTWARE OR THE USE OR OTHER DEALINGS IN THE SOFTWARE.
%----------------------------------------------------------------------%
